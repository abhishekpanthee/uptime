% COMPREHENSIVE LATEX GUIDE FOR STUDENTS

\chapter{COMPREHENSIVE \LaTeX~GUIDE}
This chapter provides a comprehensive guide to using this \LaTeX template effectively. It covers essential \LaTeX commands, formatting guidelines, and practical examples with rendered output to help you write high-quality technical documents following \gls{ieee} standards.

\section*{Template Philosophy}

This template is designed with three core principles:

\begin{itemize}
    \item \textbf{IEEE Compliance}: All formatting adheres to \gls{ieee} Transactions standards for academic publications
    \item \textbf{Automation}: Cross-references, numbering, and bibliography are handled automatically
    \item \textbf{Separation of Concerns}: Content is separated from formatting---you focus on writing, the template handles presentation
\end{itemize}

The template leverages modern \LaTeX packages (\texttt{newtxtext}, \texttt{newtxmath}, \texttt{cleveref}, \texttt{glossaries}, \texttt{booktabs}, \texttt{algorithmicx}) to produce professional, publication-ready documents.

\section*{Quick Start Guide}

\subsection*{Initial Setup}

\begin{enumerate}
    \item \textbf{Install \LaTeX Distribution}:
          \begin{itemize}
              \item Windows: MiKTeX or TeX Live
              \item macOS: MacTeX
              \item Linux: TeX Live (via package manager)
          \end{itemize}

    \item \textbf{Configure Your Information}:
          Edit \texttt{vars.tex} with your personal details, project title, supervisors, and institutional information.

    \item \textbf{Compilation Sequence}:
          To properly generate all references, bibliography, and glossaries, compile in this order:
          \begin{verbatim}
    pdflatex main.tex
    bibtex main
    makeglossaries main
    pdflatex main.tex
    pdflatex main.tex
    \end{verbatim}

          Most \LaTeX editors (TeXstudio, Overleaf, VS Code with \LaTeX Workshop) handle this automatically. Using glossaries package on VS Code requires setting up shell escape.

    \item \textbf{Project Structure}:
          \begin{verbatim}
    main.tex              - Main document file
    vars.tex              - Your personal information
    thapathaliece.cls     - Template class (do not modify)
    references.bib        - Bibliography database
    src/
      frontmatter/        - Abstract, acknowledgments, etc.
      chapters/           - Your chapter content files
      backmatter/         - Appendices, supplementary material
      images/             - All figures and diagrams
    \end{verbatim}
\end{enumerate}

\section*{Essential \LaTeX Concepts}

\subsection*{Document Structure}

Each chapter file should follow this structure:

\begin{verbatim}
\chapter{CHAPTER TITLE}

Introduction paragraph for the chapter...

\section{Section Title}

Content for this section...

\subsection{Subsection Title}

Detailed content...

\subsubsection{Subsubsection Title}

Fine-grained details...
\end{verbatim}

\subsection*{Text Formatting}

Basic text formatting commands:

\begin{verbatim}
\textbf{bold text}
\textit{italic text}
\texttt{monospace/code text}
\underline{underlined text}
\emph{emphasized text}
\end{verbatim}

\textbf{Rendered examples}:
\begin{itemize}
    \item \textbf{bold text} for emphasis
    \item \textit{italic text} for terms and titles
    \item \texttt{monospace text} for code, filenames, and commands
    \item \underline{underlined text} (use sparingly)
    \item \emph{emphasized text} (context-aware emphasis)
\end{itemize}

\subsection*{Lists}

Three types of lists are available:

\begin{verbatim}
\begin{itemize}
    \item First bullet point
    \item Second bullet point
    \item Third bullet point
\end{itemize}

\begin{enumerate}
    \item First numbered item
    \item Second numbered item
    \item Third numbered item
\end{enumerate}

\begin{description}
    \item[Term 1] Definition of term 1
    \item[Term 2] Definition of term 2
\end{description}
\end{verbatim}

\section*{Mathematical Notation}

\subsection*{Inline vs Display Math}

\LaTeX provides two modes for mathematics:

\begin{verbatim}
Inline math: $E = mc^2$ appears in text flow.

Display math:
\[
E = mc^2
\]
appears centered on its own line.
\end{verbatim}

\textbf{Rendered}: Inline math like $E = mc^2$ flows with text, while display math is centered:
\[
    E = mc^2
\]

\subsection*{Numbered Equations}

For important equations that need referencing:

\begin{verbatim}
\begin{equation}
    f(x) = \sum_{i=1}^{n} w_i \cdot x_i + b
    \label{eq:linear_model}
\end{equation}

Reference using: \cref{eq:linear_model}
\end{verbatim}

\textbf{Rendered}:
\begin{equation}
    f(x) = \sum_{i=1}^{n} w_i \cdot x_i + b
    \label{eq:linear_example}
\end{equation}

where $f(x)$ is the output, $w_i$ are weights, $x_i$ are inputs, and $b$ is the bias term. Reference as: \cref{eq:linear_example}.

\subsection*{IEEE Mathematical Notation}

Follow \gls{ieee} conventions for mathematical symbols:

\begin{verbatim}
Variables (italic):     $x$, $y$, $n$
Vectors (bold):         $\vect{v}$, $\vect{x}$
Matrices (bold):        $\matr{A}$, $\matr{X}$
Sets (calligraphic):    $\set{S}$, $\set{P}$
Functions (roman):      $\sin(x)$, $\log(n)$
\end{verbatim}

\textbf{Rendered}: Variables $x$, $y$, $n$; vectors $\vect{v}$, $\vect{x}$; matrices $\matr{A}$, $\matr{X}$; sets $\set{S}$, $\set{P}$; functions $\sin(x)$, $\log(n)$.

\section*{Figures and Images}

\subsection*{Including Figures}

Basic figure syntax following \gls{ieee} standards:

\begin{verbatim}
\begin{figure}[htbp]
    \centering
    \includegraphics[width=0.8\textwidth]{src/images/diagram.png}
    \caption{System architecture showing main components.}
    \label{fig:architecture}
\end{figure}

Reference using: \cref{fig:architecture}
\end{verbatim}

\textbf{Rendered output}:

\begin{figure}[htbp]
    \centering
    \includegraphics[width=0.6\textwidth]{src/images/logo.png}
    \caption{System architecture showing main components and their interactions.}
    \label{fig:example_architecture}
\end{figure}

As shown in \cref{fig:example_architecture}, the architecture demonstrates proper figure formatting with caption below.

\textbf{Key points}:
\begin{itemize}
    \item \texttt{[htbp]} specifies placement: here, top, bottom, page
    \item \texttt{\textbackslash centering} centers the figure
    \item Width can be: \texttt{0.5\textbackslash textwidth}, \texttt{10cm}, \texttt{width=\textbackslash textwidth}
    \item Caption appears \textbf{below} figure (\gls{ieee} standard)
    \item Always use \texttt{\textbackslash label} for cross-referencing
    \item Use \texttt{\textbackslash cref} for automatic ``Fig.'' prefix
\end{itemize}

\subsection*{Subfigures}

For multiple related figures:

\begin{verbatim}
\begin{figure}[htbp]
    \centering
    \begin{subfigure}[b]{0.45\textwidth}
        \includegraphics[width=\textwidth]{image1.png}
        \caption{First subfigure}
        \label{fig:sub1}
    \end{subfigure}
    \hfill
    \begin{subfigure}[b]{0.45\textwidth}
        \includegraphics[width=\textwidth]{image2.png}
        \caption{Second subfigure}
        \label{fig:sub2}
    \end{subfigure}
    \caption{Overall caption for both subfigures.}
    \label{fig:combined}
\end{figure}
\end{verbatim}

Reference individual subfigures: \texttt{\textbackslash cref\{fig:sub1\}} or the entire figure: \texttt{\textbackslash cref\{fig:combined\}}.

\textbf{Rendered output}:

\begin{figure}[htbp]
    \centering
    \begin{subfigure}[b]{0.45\textwidth}
        \includegraphics[width=\textwidth]{src/images/logo.png}
        \caption{Training accuracy over epochs}
        \label{fig:demo_train}
    \end{subfigure}
    \hfill
    \begin{subfigure}[b]{0.45\textwidth}
        \includegraphics[width=\textwidth]{src/images/logo.png}
        \caption{Validation accuracy over epochs}
        \label{fig:demo_val}
    \end{subfigure}
    \caption{Model performance during training showing both training and validation metrics.}
    \label{fig:demo_performance}
\end{figure}

The subfigures (\cref{fig:demo_train,fig:demo_val}) show different aspects of the same experiment, while \cref{fig:demo_performance} references the entire figure.

\section*{Tables}

\subsection*{IEEE-Compliant Tables}

\textbf{Critical}: Use \texttt{booktabs} package rules only. toprule, midrule and bottom rule. Never use vertical lines (for column separation) or \texttt{\textbackslash hline}. Look at this example! The top line and bottom lines are thick while the middle line is thin. This consistency is required throughout the report.
Always label and cross reference all tables in the document.

\begin{table}[htbp]
    \centering
    \caption{Experimental parameters and their values.}
    \label{tab:parameters}
    \begin{tabular}{lcc}
        \toprule
        \textbf{Parameter} & \textbf{Value} & \textbf{Unit} \\
        \midrule
        Population Size    & 100            & individuals   \\
        Generations        & 500            & iterations    \\
        Mutation Rate      & 0.05           & probability   \\
        \bottomrule
    \end{tabular}
\end{table}


\begin{verbatim}
\begin{table}[htbp]
    \centering
    \caption{Experimental parameters and their values.}
    \label{tab:parameters}
    \begin{tabular}{lcc}
        \toprule
        \textbf{Parameter} & \textbf{Value} & \textbf{Unit} \\
        \midrule
        Population Size    & 100            & individuals   \\
        Generations        & 500            & iterations    \\
        Mutation Rate      & 0.05           & probability   \\
        \bottomrule
    \end{tabular}
\end{table}

Reference using: \cref{tab:parameters}
\end{verbatim}

\textbf{Key points}:
\begin{itemize}
    \item Caption appears \textbf{above} table (\gls{ieee} standard)
    \item Use \texttt{\textbackslash toprule}, \texttt{\textbackslash midrule}, \texttt{\textbackslash bottomrule} only
    \item Column alignment: \texttt{l} (left), \texttt{c} (center), \texttt{r} (right)
    \item Bold headers: \texttt{\textbackslash textbf\{Header\}}
    \item Never use \texttt{|} for vertical lines
\end{itemize}

\section*{Code and Pseudocode in Academic Reports}

\subsection*{IEEE Standard for Formal Reports}

\textbf{Important Note}: For formal academic reports following \gls{ieee} standards, pseudocode using the \texttt{algorithmicx} package is \textbf{strongly preferred} over direct code embedding.

\subsubsection*{Why Pseudocode is Preferred}

\begin{itemize}
    \item \textbf{Language-agnostic}: Focuses on logic rather than implementation details
    \item \textbf{Professional presentation}: Consistent with academic publication standards
    \item \textbf{Better readability}: Semantic structure with clear control flow
    \item \textbf{Space efficient}: Avoids lengthy implementation details
    \item \textbf{IEEE compliant}: Follows academic formatting conventions
\end{itemize}

\subsubsection*{When Direct Code May Be Acceptable}

Direct code listings are generally \textbf{discouraged} in formal academic reports. However, they may be appropriate in:

\begin{itemize}
    \item \textbf{Appendices}: For reference or reproducibility
    \item \textbf{Technical documentation}: Implementation guides or manuals
    \item \textbf{Software-focused papers}: When code is the primary contribution
    \item \textbf{API demonstrations}: Showing specific library usage
\end{itemize}

\subsection*{Inline Code References}

For mentioning function names, variables, or commands in text:

\begin{verbatim}
Use the \texttt{calculate\_fitness()} function to evaluate solutions.
Set the \texttt{population\_size} parameter to 100.
\end{verbatim}

\textbf{Rendered}: Use the \texttt{calculate\_fitness()} function to evaluate solutions. Set the \texttt{population\_size} parameter to 100.

\section*{Algorithms and Pseudocode}

\subsection*{Algorithm Environment}

For formal algorithms, use \texttt{algorithmicx} with \texttt{algpseudocode}:

\begin{verbatim}
\begin{algorithm}[htbp]
\caption{Genetic Algorithm for Optimization}
\label{alg:genetic}
\begin{algorithmic}[1]
\Require Population size $N$, generations $G$
\Ensure Optimized solution $\mathbf{x}^*$
\State Initialize population $P \gets \text{random}(N)$
\For{$g = 1$ to $G$}
    \State Evaluate fitness for all individuals in $P$
    \State $P' \gets \text{selection}(P)$
    \State $P'' \gets \text{crossover}(P')$
    \State $P \gets \text{mutation}(P'')$
\EndFor
\State \Return best individual from $P$
\end{algorithmic}
\end{algorithm}

Reference using: \cref{alg:genetic}
\end{verbatim}

\textbf{Rendered output}:

\begin{algorithm}[htbp]
    \caption{Genetic Algorithm for Optimization}
    \label{alg:genetic_rendered}
    \begin{algorithmic}[1]
        \Require Population size $N$, generations $G$
        \Ensure Optimized solution $\mathbf{x}^*$
        \State Initialize population $P \gets \text{random}(N)$
        \For{$g = 1$ to $G$}
        \State Evaluate fitness for all individuals in $P$
        \State $P' \gets \text{selection}(P)$
        \State $P'' \gets \text{crossover}(P')$
        \State $P \gets \text{mutation}(P'')$
        \EndFor
        \State \Return best individual from $P$
    \end{algorithmic}
\end{algorithm}

\textbf{Key commands}:
\begin{itemize}
    \item \texttt{\textbackslash State}: Single statement
    \item \texttt{\textbackslash If....\textbackslash EndIf}: Conditional blocks
    \item \texttt{\textbackslash For....\textbackslash EndFor}: Loops
    \item \texttt{\textbackslash While....\textbackslash EndWhile}: While loops
    \item \texttt{\textbackslash Require}: Input requirements
    \item \texttt{\textbackslash Ensure}: Output guarantee
    \item \texttt{\textbackslash Return}: Return statement
\end{itemize}

\section*{Citations and References}

\subsection*{Adding References}

Add entries to \texttt{references.bib} in BibTeX format:

\begin{verbatim}
@article{smith2020machine,
    author  = {Smith, John and Doe, Jane},
    title   = {Machine Learning for Optimization},
    journal = {IEEE Transactions on Neural Networks},
    year    = {2020},
    volume  = {31},
    number  = {5},
    pages   = {1234--1245}
}
\end{verbatim}

\subsection*{Citing References}

Use \texttt{\textbackslash cite} command:

\begin{verbatim}
Machine learning has shown promising results \cite{smith2020machine}.
Multiple citations can be combined \cite{ref1,ref2,ref3}.
\end{verbatim}

\textbf{Rendered}: Citations appear as numbers in square brackets: [1], [2--5].

\section*{Acronyms and Glossaries}

\subsection*{Defining Acronyms}

Add definitions to \texttt{src/frontmatter/abbreviations.tex}:

\begin{verbatim}
\newacronym{ml}{ML}{Machine Learning}
\newacronym{ai}{AI}{Artificial Intelligence}
\newacronym{nn}{NN}{Neural Network}
\end{verbatim}

\subsection*{Using Acronyms}

\begin{verbatim}
\gls{ml}      - First use: Machine Learning (ML)
              - Later uses: ML
\glspl{nn}    - Plural form: NNs
\Gls{ai}      - Capitalized: Artificial Intelligence (AI)
\acrshort{ml} - Always shows: ML
\acrlong{ml}  - Always shows: Machine Learning
\end{verbatim}

\textbf{Benefits}:
\begin{itemize}
    \item Automatic expansion on first use
    \item Consistent abbreviation usage throughout document
    \item Automatic generation of abbreviations list
    \item Prevents undefined acronym errors
\end{itemize}

\section*{Cross-Referencing}

\subsection*{Modern Cross-References with Cleveref}

This template uses \texttt{cleveref} package for intelligent cross-referencing:

\begin{verbatim}
\cref{fig:arch}        → Fig. 1.1
\cref{tab:results}     → Table 2.3
\cref{eq:fitness}      → Eq. (3.4)
\cref{alg:nsga}        → Algorithm 4.1
\cref{sec:method}      → Section 5.2
\cref{chap:intro}      → Chapter 1

Multiple references:
\cref{fig:a,fig:b,fig:c}    → Fig. 1.1, 1.2, and 1.3
\cref{tab:x,eq:y}           → Table 2.1 and Eq. (3.5)
\end{verbatim}

\textbf{Advantages over manual referencing}:
\begin{itemize}
    \item Automatic prefix (Fig., Table, Eq., etc.)
    \item Automatic number updating when content changes
    \item Intelligent grouping of multiple references
    \item Hyperlinks to referenced elements (in PDF)
\end{itemize}

\section*{Special Characters and Typography}

\subsection*{Dashes}

\LaTeX distinguishes between three types of dashes:

\begin{verbatim}
Hyphen:     state-of-the-art (single -)
En-dash:    pages 10--20 (double --)
Em-dash:    results---as shown---improved (triple ---)
\end{verbatim}

\textbf{Rendered}: state-of-the-art, pages 10--20, results---as shown---improved.

\subsection*{Quotation Marks}

Use proper typographic quotes:

\begin{verbatim}
``double quotes''
`single quotes'
\end{verbatim}

\textbf{Rendered}: ``double quotes'' and `single quotes'.

\textbf{Never} use straight quotes copied from Word or text editors.

\subsection*{Special Symbols}

Common special characters require backslashes:

\begin{verbatim}
\%    - Percent sign
\$    - Dollar sign
\&    - Ampersand
\_    - Underscore
\#    - Hash/pound
\{    - Left brace
\}    - Right brace
\~{}  - Tilde
\end{verbatim}

\section*{Best Practices and Workflow}

\subsection*{File Organization}

\begin{itemize}
    \item \textbf{One chapter per file}: Keep chapters in separate files for easier management
    \item \textbf{Descriptive filenames}: Use names like \texttt{methodology.tex}, \texttt{results.tex}
    \item \textbf{Image organization}: Store images in \texttt{src/images/} with subdirectories if needed
    \item \textbf{Version control}: Use Git to track changes and collaborate
\end{itemize}

\subsection*{Compilation Tips}

\begin{itemize}
    \item \textbf{Clean auxiliary files}: Regularly delete \texttt{.aux}, \texttt{.log}, \texttt{.toc} files if errors persist
    \item \textbf{Full recompile}: Run complete sequence (pdflatex → bibtex → makeglossaries → pdflatex × 2)
    \item \textbf{Check errors}: Read error messages carefully---they indicate file and line number
    \item \textbf{Incremental compilation}: Compile frequently to catch errors early
\end{itemize}

\subsection*{Common Mistakes to Avoid}

\begin{itemize}
    \item \textbf{UTF-8 characters}: Avoid smart quotes, em-dashes, and special characters copied from Word
    \item \textbf{Unclosed braces}: Every \texttt{\{} must have matching \texttt{\}}
    \item \textbf{Missing labels}: Always add \texttt{\textbackslash label} to figures, tables, equations, and algorithms
    \item \textbf{Unreferenced elements}: Every figure, table, and equation must be referenced in text
    \item \textbf{Manual numbering}: Never type ``Fig. 1'' manually---use \texttt{\textbackslash cref}
    \item \textbf{Vertical lines in tables}: Use \texttt{booktabs} rules, not \texttt{|} or \texttt{\textbackslash hline}
\end{itemize}

\subsection*{Quality Checklist}

Before final submission, verify:

\begin{itemize}
    \item[$\square$] Document compiles without errors or warnings
    \item[$\square$] All figures, tables, and equations are referenced in text
    \item[$\square$] All citations appear in bibliography
    \item[$\square$] Abbreviations are defined and used consistently
    \item[$\square$] Cross-references show numbers (no ``??'')
    \item[$\square$] Table of contents has correct page numbers
    \item[$\square$] All code listings have captions and proper formatting
    \item[$\square$] Mathematical notation follows \gls{ieee} conventions
    \item[$\square$] No UTF-8 encoding errors in bibliography
    \item[$\square$] PDF bookmarks and hyperlinks work correctly
\end{itemize}

\section*{Advanced Features}

\subsection*{Custom Commands}

The template provides custom commands for convenience:

\begin{verbatim}
\vect{v}      - Bold vector notation
\matr{A}      - Bold matrix notation
\set{S}       - Calligraphic set notation
\ie           - i.e.,
\eg           - e.g.,
\etc          - etc.
\end{verbatim}

\subsection*{Units with siunitx}

For proper unit formatting:

\begin{verbatim}
\SI{5}{\meter}
\SI{100}{\kilo\gram}
\SI{3.5}{\giga\hertz}
\SIrange{10}{20}{\celsius}
\end{verbatim}

\textbf{Rendered}: 5 m, 100 kg, 3.5 GHz, 10--20°C.

\section*{Getting Help}

\subsection*{Resources}

\begin{itemize}
    \item \textbf{This document}: Read all guideline chapters thoroughly
    \item \textbf{LaTeX Stack Exchange}: \texttt{tex.stackexchange.com} for specific questions
    \item \textbf{Overleaf Documentation}: Comprehensive \LaTeX tutorials
    \item \textbf{Template class file}: See \texttt{thapathaliece.cls} for implementation details
\end{itemize}

\subsection*{Common Issues}

\begin{description}
    \item[Bibliography not appearing] Run: pdflatex → bibtex → pdflatex × 2
    \item[Acronyms not expanding] Run: pdflatex → makeglossaries → pdflatex × 2
    \item[Undefined references (??)] Compile multiple times until stable
    \item[Missing images] Check file paths and extensions (.png, .png, .jpg)
    \item[UTF-8 errors] Replace special characters in .bib file with \LaTeX equivalents
\end{description}



% ============================================================================
% PART II: DETAILED IEEE WRITING AND FORMATTING GUIDELINES
% ============================================================================

\section*{WRITING GUIDELINES}

This section provides comprehensive guidance on technical writing following IEEE standards and best practices for academic publications \cite{ieee2022author, alley2018craft}.


\subsection*{Avoiding Common Mistakes}

\subsubsection*{Grammar and Style}

\begin{itemize}
    \item \textbf{Avoid}: ``The data was analyzed'' → \textbf{Use}: ``The data were analyzed'' (data is plural)
    \item \textbf{Avoid}: ``very unique'', ``more optimal'' → \textbf{Use}: ``unique'', ``optimal'' (absolute adjectives)
    \item \textbf{Avoid}: ``In this paper'' → \textbf{Use}: ``In this report'' or ``In this work''
    \item \textbf{Avoid}: Contractions (don't, can't) → \textbf{Use}: Full forms (do not, cannot)
\end{itemize}

\subsubsection*{Technical Writing}

\begin{itemize}
    \item Define terms before using them extensively
    \item Maintain consistency in terminology throughout
    \item Use specific numbers rather than vague terms: ``15\% improvement'' not ``significant improvement''
    \item Explain acronyms at first use, even if well-known in the field
    \item Cross-reference related sections appropriately using \texttt{\textbackslash cref\{\}} commands
\end{itemize}


\section*{COMPILING AND TROUBLESHOOTING}

This section provides practical guidance on compiling your report and resolving common issues.

\subsection*{Compilation Process}

\subsubsection*{Standard Compilation Sequence}

To properly compile this report with all references, use this sequence:

\begin{enumerate}
    \item \textbf{pdf\LaTeX}: First pass to generate auxiliary files
    \item \textbf{bibtex}: Process bibliography references
    \item \textbf{makeglossaries}: Generate abbreviations list
    \item \textbf{pdf\LaTeX}: Second pass to include bibliography
    \item \textbf{pdf\LaTeX}: Third pass to resolve all cross-references
\end{enumerate}

\subsubsection*{Command Line Compilation}

Open terminal in report directory and run:

\begin{verbatim}
pdflatex main.tex
bibtex main
makeglossaries main
pdflatex main.tex
pdflatex main.tex
\end{verbatim}

\subsubsection*{Using \LaTeX Editors}

Most \LaTeX editors have "Build" buttons that handle the sequence automatically:

\begin{itemize}
    \item \textbf{TeXstudio}: Press F5 or click "Build \& View"
    \item \textbf{Overleaf}: Compiles automatically on save
    \item \textbf{VS Code}: With \LaTeX Workshop extension, use "Build \LaTeX project"
\end{itemize}

\subsection*{Common Compilation Errors}

\subsubsection*{Missing Package Errors}

\textbf{Error}: \texttt{! \LaTeX Error: File 'package.sty' not found}

\textbf{Solution}: Install the missing package:
\begin{itemize}
    \item TeX Live: \texttt{tlmgr install package}
    \item MiKTeX: Will prompt to install automatically
    \item Or install full \LaTeX distribution
\end{itemize}

\subsubsection*{Undefined Reference Warnings}

\textbf{Warning}: \texttt{\LaTeX Warning: Reference 'fig:example' undefined}

\textbf{Solution}: Run pdf\LaTeX multiple times (2--3) after adding new labels. References require multiple passes to resolve.

\subsubsection*{Citation Undefined}

\textbf{Warning}: \texttt{Citation 'smith2020' undefined}

\textbf{Solution}:
\begin{itemize}
    \item Check that citation key exists in \texttt{references.bib}
    \item Run bibtex step
    \item Run pdflatex again
\end{itemize}

\subsubsection*{Glossary Not Appearing}

\textbf{Problem}: Abbreviations list is empty or doesn't appear

\textbf{Solution}:
\begin{itemize}
    \item Ensure you've used \texttt{\textbackslash gls\{\}} commands in text
    \item Run \texttt{makeglossaries} command
    \item Run pdf\LaTeX again
    \item Glossaries only shows acronyms actually used in document
\end{itemize}

\subsubsection*{Font Errors}

\textbf{Error}: Font shape warnings or missing font files

\textbf{Solution}:
\begin{itemize}
    \item Ensure newtx package is installed: \texttt{tlmgr install newtx}
    \item Update \LaTeX distribution to latest version
    \item Run \texttt{updmap} or \texttt{updmap-sys} to refresh font database
\end{itemize}

\subsection*{File Organization Issues}

\subsubsection*{File Not Found Errors}

\textbf{Error}: \texttt{! \LaTeX Error: File 'src/chapters/intro.tex' not found}

\textbf{Solution}:
\begin{itemize}
    \item Verify file exists in correct directory
    \item Check for typos in filename or path
    \item Use forward slashes (/) not backslashes (\textbackslash) in paths
    \item Ensure no special characters in filenames
\end{itemize}

\subsubsection*{Image Not Found}

\textbf{Error}: \texttt{! Package pdftex.def Error: File 'img/figure.png' not found}

\textbf{Solution}:
\begin{itemize}
    \item Check that image file exists in specified path
    \item Verify file extension is correct (.pdf, .png, .jpg)
    \item Use relative paths from main .tex file location
    \item Ensure image is in supported format
\end{itemize}

\subsection*{Managing Large Documents}

\subsubsection*{Compilation Time}

For large theses with many figures:

\begin{itemize}
    \item Use \texttt{\textbackslash includeonly\{chap/chapter\_name\}} to compile only specific chapters during editing
    \item Keep this in main .tex file, comment out when not needed
    \item Final compilation should include all chapters
\end{itemize}

Example:
\begin{verbatim}
% Uncomment to compile only specific chapters:
% \includeonly{chap/methodology,chap/results}
\end{verbatim}

\subsubsection*{Draft Mode}

For faster compilation during editing:

\begin{verbatim}
\documentclass[draft]{scrreprt}
\end{verbatim}

Draft mode:
\begin{itemize}
    \item Shows boxes instead of images
    \item Marks overfull boxes clearly
    \item Compiles faster
    \item Remember to remove \texttt{draft} option for final version
\end{itemize}

\subsection*{Quality Checks}

\subsubsection*{Before Final Submission}

Run these checks:

\begin{enumerate}
    \item \textbf{Complete compilation}: Full pdf\LaTeX → bibtex → makeglossaries → pdf\LaTeX × 2
    \item \textbf{No warnings}: Address all \LaTeX warnings in log
    \item \textbf{References check}: All citations appear in bibliography
    \item \textbf{Cross-references}: No "??" in document
    \item \textbf{Figures}: All figures appear correctly, not missing
    \item \textbf{Page numbers}: TOC page numbers match actual pages
    \item \textbf{Spelling}: Run spell checker
    \item \textbf{Formatting}: Consistent throughout document
\end{enumerate}

\subsubsection*{PDF Quality}

Verify final PDF:

\begin{itemize}
    \item \textbf{File size}: Reasonable (< 50 MB typically)
    \item \textbf{Hyperlinks}: Internal links work correctly
    \item \textbf{Bookmarks}: Chapter bookmarks appear in PDF viewer
    \item \textbf{Fonts}: All fonts embedded properly
    \item \textbf{Images}: Clear and readable at 100\% zoom
\end{itemize}



\vspace{1em}
\noindent\textit{Best wishes for your writing,} \\
Krishna Acharya,\\
