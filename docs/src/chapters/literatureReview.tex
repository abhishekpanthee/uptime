\chapter{LITERATURE REVIEW}


WRITING GUIDELINES FOR LITERATURE REVIEW CHAPTER

The Literature Review critically analyzes existing research, technologies, and solutions related to your project. It is NOT merely a summary of papers but rather a synthesis that compares approaches, identifies gaps, and justifies your project's novelty. This chapter should demonstrate your understanding of the field and position your work within the broader research landscape.

GENERAL ORGANIZATION PRINCIPLES:
\begin{itemize}
    \item Structure by themes or technologies, NOT paper-by-paper summaries
    \item Write in reverse chronological order within each theme
    \item Compare and contrast different approaches using tables
    \item Identify trends and gaps in existing research
    \item Build argument for why your work is needed
    \item Cite comprehensively but critically (15-30 high-quality references)
    \item Each paragraph should relate reviewed work to YOUR project
\end{itemize}

HOW TO WRITE - HARDWARE PROJECTS:

Review existing hardware systems addressing similar problems. Describe their architectures, components, performance characteristics, and limitations. Compare different: Sensor technologies, Microcontroller platforms (Arduino, Raspberry Pi, ESP32, STM32), Communication protocols (UART, I2C, SPI, WiFi, Bluetooth), Power management strategies.

For ML-enhanced hardware: Review edge AI approaches, Model optimization techniques (quantization, pruning), Inference engines (TensorFlow Lite, ONNX Runtime), Hardware accelerators (GPUs, TPUs, Neural Processing Units).

Identify gaps: existing systems too expensive, excessive power consumption, lack functionality, cannot operate in specific environments, fail to leverage modern ML techniques.

Include 15-20 high-quality references from IEEE journals, conferences, and technical sources.

HOW TO WRITE - ML/SOFTWARE PROJECTS:

Organize around key themes: Application domain background, Relevant algorithms or architectures (CNNs, Transformers, ensemble methods), Datasets commonly used, Evaluation methodologies, Deployment considerations.

For ML projects: Review foundational algorithms/architectures. Discuss strengths, weaknesses, computational requirements, and applicability. Compare different approaches (traditional ML vs. deep learning, supervised vs. unsupervised). Use tables to compare model architectures, performance metrics, dataset sizes, and computational costs. Critically evaluate methodologies.

For software projects: Review existing systems, frameworks, or applications. Discuss architectural patterns (MVC, microservices, serverless), technology stacks, scalability approaches, UI paradigms. Compare commercial vs. open-source alternatives.

Identify research gaps: existing models don't generalize to your domain, available software lacks features, current solutions don't scale, insufficient research comparing approaches.

Aim for 20-30 high-quality references from top-tier venues (NeurIPS, ICML, CVPR, ACL, SIGMOD) and industry technical reports.

STRUCTURE SUGGESTIONS:
section Related Hardware/Software Systems, section Machine Learning Approaches (if applicable), section Datasets and Benchmarks, section Evaluation Methodologies, section Research Gaps and Our Contribution

LaTeX quick refs: Cite with cite key, Use label and reference via Section ref, Use comparison tables with label and reference via Table ref